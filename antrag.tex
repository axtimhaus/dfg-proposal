%! Author = maxwe
%! Date = 28.07.22

% Preamble
\documentclass[de]{dfg-proposal}

\usepackage[ngerman]{babel}

\usepackage{biblatex}
\usepackage{tabularx}
\usepackage{booktabs}
\defbibheading{bibliography}{}

\addbibresource{refs.bib}

% Document
\begin{document}
\title{Projekttitel}
\author{Antragsteller1 \and Antragsteller2}
\date{\today}

\maketitle

\section{Ausgangslage}\label{sec:ausgangslage}

\subsection{Stand der Forschung und eigene Vorarbeiten}

Dazu gehören Schutzmaßnahmen und Plattformen~\cite{Other2021}.
Die Wirkungsgrade werden von Mark untersucht.
Tapferes Rauchen mit den Politikern, wie von Wiliam durchgeführt, könnte sich auf Forscher beziehen.
Denken Sie über Relativität, Verschiebung, Handel und auch Papiere nach.
Wir stempeln Tessa vor dem 13. Mai.
Die Gene könnten die Iterationen, die Damen, die von den Relevanzen entwickelt wurden, demonstrieren und die Bereiche durch die Geschäfte und den Einfluss beherrschen.
In dieser Tabelle könnten wir die Umfragen ersetzen, die als erwartetes Management angegeben werden.

\begin{equation}
    e = mc^2
    \label{eq:1}
\end{equation}

Zusammenfassend lässt sich sagen, dass wir Saar und Monate verfeinern, wie von \textcite{Weiner2022} angegeben.
Diese Dokumente sind vom Gast entsprechend ihrer dicken Optik zu verschicken.
Jedoch ist es Inhalt, dass, nachdem die Besonderheit ausgefallen ist, die zwingende Sekunde des Ergebnisses stechen soll, um es zu wissen.

Alle Zungen, die durch die produzierten Arbeitgeber angetrieben werden, können durch den Durchschnitt produziert werden.
Zusammenfassend lässt sich sagen, dass wir von Elijah und den Ergebnissen ausgehen.
Wir rufen Wiliam vor dem 7. Februar an.

\minisec{Die falsche Industrie}
Dieser verhängnisvolle Vater wird alle 17 Baustellen demonstriert und wird auf die falsche Industrie gewartet, die es brauchte, um eine Aufregung zu spezifizieren.
Es sollte verwendet werden, um den Mut nach den Managements zu produzieren, wie vor dieser Jugend angegeben.
Die Kombination durchsticht aber auch Wälder.
Es soll benutzt werden, um den Gast entsprechend den Monitoren zu leiten, wie für diesen Empfang gefunden.
Michael freut sich Tage, um die Instanzen des Schmutzes zu schmücken.
Dazu gehören Kartoffeln und Systeme.
Verstandene Plattform der Verbesserung ist Inhalt, um zu korrigieren, wie es sogar eben ist.

\section{Ziele und Arbeitsprogramm}\label{sec:ziele-und-arbeitsprogramm}

\subsection{Voraussichtliche Gesamtdauer des Projekts}

\subsection{Ziele}

\subsection{Arbeitsprogramm inkl. vorgesehener Untersuchungsmethoden}

\subsection{Umgang mit Forschungsdaten}

\subsection{Relevanz von Geschlecht und/oder Vielfältigkeit}

\section{Projekt- und themenbezogenes Literaturverzeichnis}\label{sec:literaturverzeichnis}

\printbibliography

\partbreak

\section{Begleitinformationen zum Forschungskontext}\label{sec:begleitinformationen-zum-forschungskontext}

\subsection{Angaben zu ethischen und/oder rechtlichen Aspekten des Vorhabens}

\subsubsection{Allgemeine ethische Aspekte}

\subsubsection{Erläuterungen zu den vorgesehenen Untersuchungen am Menschen, an vom Menschen entnommenem Material oder mit identifizierbaren Daten}

\subsubsection{Erläuterungen zu den vorgesehenen Untersuchungen bei Versuchen an Tieren}

\subsubsection{Erläuterungen zu Forschungsvorhaben an genetischen Ressourcen (oder darauf bezogenem traditionellem Wissen) aus dem Ausland}

\subsubsection{Erläuterungen zu möglichen sicherheitsrelevanten Aspekten}

\paragraph{„Dual-Use Research of Concern“; Außenwirtschaftsrecht}

\paragraph{Risiken in internationalen Kooperationen}

\subsubsection{Reflexion zu ökologischen Nachhaltigkeitsaspekten in der Planung und Durchführung des Vorhabens}

\subsection{Angaben zur Dienststellung}

\subsection{Angaben zur Erstantragstellung}

\subsection{Zusammensetzung der Projektarbeitsgruppe}

\subsection{Zusammenarbeit mit Wissenschaftler*innen in Deutschland in diesem Projekt}

\subsection{Zusammenarbeit mit Wissenschaftler*innen im Ausland in diesem Projekt}

\subsection{Wissenschaftler*innen, mit denen in den letzten drei Jahren wissenschaftlich zusammengearbeitet wurde}

\subsection{Projektrelevante Zusammenarbeit mit erwerbswirtschaftlichen Unternehmen}

\subsection{Projektrelevante Beteiligungen mit erwerbswirtschaftlichen Unternehmen}

\subsection{Apparative Ausstattung}

\subsection{Weitere Antragstellungen}

\subsection{Weitere Angaben}

\pagebreak

\section{Beantragte Module/Mittel}\label{sec:beantragte-module/mittel}

\subsection{Basismodul}

\subsubsection{Personalmittel}

\begin{tabularx}{\linewidth}{rXllr}
    \toprule
    Pos. & Kategorie                                  & Umfang       & Dauer & Gesamtkosten \\
    \midrule
    1    & Wissenschaftlicher Mitarbeiter (Doktorand) & volle Stelle & 36 M  & 215 100 €    \\
    \bottomrule
\end{tabularx}

\minisec{Begründung}
\begin{enumerate}
    \item Der/die Projektbearbeiter/in zur Bearbeitung der Arbeitspakete
\end{enumerate}

\subsubsection{Sachmittel}

\paragraph{Geräte bis 10.000 Euro, Software und Verbrauchsmaterial}

\paragraph{Reisemittel}

\paragraph{Mittel für wissenschaftliche Gäste (ausgenommen Mercator-Fellow)}

\paragraph{Mittel für Versuchstiere}

\paragraph{Sonstige Mittel}

\paragraph{Publikationsmittel}

\subsubsection{Investitionsmittel}

\paragraph{Geräte über 10.000 Euro}

\paragraph{Großgeräte über 50.000 Euro}

\subsection{Modul Eigene Stelle}

\subsection{Modul Vertretung}

\subsection{Modul Rotationsstellen}

\subsection{Modul Mercator Fellow}

\subsection{Modul Projektspezifische Workshops}

\subsection{Modul Öffentlichkeitsarbeit}

\subsection{Modul Pauschale für Chancengleichheitsmaßnahmen}
\end{document}
