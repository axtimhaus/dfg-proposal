%! Author = maxwe
%! Date = 28.07.22

% Preamble
\documentclass[de]{dfg-proposal}
\usepackage{biblatex}


% Document
\begin{document}
    \title{Projekttitel}
    \author{Antragsteller1 \and Antragsteller2}


    \maketitle


    \section{Ausgangslage}\label{sec:ausgangslage}

        \subsection{Stand der Forschung und eigene Vorarbeiten}

        \subsection{Projektbezogenes Publikationsverzeichnis Ihrer Arbeiten}

            \subsubsection{Veröffentlichte Arbeiten aus Publikationsorganen mit wissenschaftlicher Qualitätssicherung, Buchveröffentlichungen sowie bereits zur Veröffentlichung angenommene, aber noch nicht veröffentlichte Arbeiten}

            \subsubsection{Andere Veröffentlichungen mit und ohne wissenschaftliche Qualitätssicherung}

            \subsubsection{Patente}


    \section{Ziele und Arbeitsprogramm}\label{sec:ziele-und-arbeitsprogramm}

        \subsection{Voraussichtliche Gesamtdauer des Projekts}

        \subsection{Ziele}

        \subsection{Arbeitsprogramm inkl. vorgesehener Untersuchungsmethoden}

        \subsection{Umgang mit Forschungsdaten}

        \subsection{Relevanz von Geschlecht und/oder Vielfältigkeit}


    \section{Literaturverzeichnis zum Stand der Forschung, zu den Zielen und dem Arbeitsprogramm}\label{sec:literaturverzeichnis}

        \printbibliography

        \partbreak


    \section{Begleitinformationen zum Forschungskontext}\label{sec:begleitinformationen-zum-forschungskontext}

        \subsection{Angaben zu ethischen und/oder rechtlichen Aspekten des Vorhabens}

            \subsubsection{Allgemeine ethische Aspekte}

            \subsubsection{Erläuterungen zu den vorgesehenen Untersuchungen am Menschen, an vom Menschen entnommenem Material oder mit identifizierbaren Daten}

            \subsubsection{Erläuterungen zu den vorgesehenen Untersuchungen bei Versuchen an Tieren}

            \subsubsection{Erläuterungen zu Forschungsvorhaben an genetischen Ressourcen (oder darauf bezogenem traditionellem Wissen) aus dem Ausland}

            \subsubsection{Erläuterungen zu möglichen sicherheitsrelevanten Aspekten („Dual-Use Research of Concern“; Außenwirtschaftsrecht)}

        \subsection{Angaben zur Dienststellung}

        \subsection{Angaben zur Erstantragstellung}

        \subsection{Zusammensetzung der Projektarbeitsgruppe}

        \subsection{Zusammenarbeit mit Wissenschaftlerinnen und Wissenschaftlern in Deutschland in diesem Projekt}

        \subsection{Zusammenarbeit mit Wissenschaftlerinnen und Wissenschaftlern im Ausland in diesem Projekt}

        \subsection{Wissenschaftlerinnen und Wissenschaftler, mit denen in den letzten drei Jahren wissenschaftlich zusammengearbeitet wurde}

        \subsection{Projektrelevante Zusammenarbeit mit erwerbswirtschaftlichen Unternehmen}

        \subsection{Apparative Ausstattung}

        \subsection{Weitere Antragstellungen}

        \subsection{Weitere Angaben}


    \section{Beantragte Module/Mittel}\label{sec:beantragte-module/mittel}

        \subsection{Basismodul}

            \subsubsection{Personalmittel}

            \subsubsection{Sachmittel}

                \paragraph{Geräte bis 10.000 Euro, Software und Verbrauchsmaterial}

                \paragraph{Reisemittel}

                \paragraph{Mittel für wissenschaftliche Gäste (ausgenommen Mercator-Fellow)}

                \paragraph{Mittel für Versuchstiere}

                \paragraph{Sonstige Mittel}

                \paragraph{Publikationsmittel}

            \subsubsection{Investitionsmittel}

                \paragraph{Geräte über 10.000 Euro}

                \paragraph{Großgeräte über 50.000 Euro}

        \subsection{Modul Eigene Stelle}

        \subsection{Modul Vertretung}

        \subsection{Modul Rotationsstellen}

        \subsection{Modul Mercator Fellow}

        \subsection{Modul Projektspezifische Workshops}

        \subsection{Modul Öffentlichkeitsarbeit}

        \subsection{Modul Pauschale für Chancengleichheitsmaßnahmen}
\end{document}